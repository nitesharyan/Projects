

%\chapter*{Abstract}

% Comment the next line if you do not want abstract to appear in table of contents
\addcontentsline{toc}{chapter}{Abstract}
\noindent
\begin{center}
\huge Abstract
\end{center}

\noindent
In this paper we had analyzed a new dataset of scholarly publications, the Microsoft Academic Graph (MAG). The MAG is a heterogeneous graph comprised of over 1.26 billion publication entities and related authors, institutions, venues and fields of study. It is also the largest publicly available dataset of citation data. As such, it is an important resource for scholarly communications research. As the dataset is assembled using automatic methods, it is important to understand its strengths and limitations, especially whether there is any noise or bias in the data, before applying it to a particular task. This article studies the characteristics of the dataset and provides a correlation analysis with other publicly available research publication datasets to answer these questions. Our results show that the citation data and publication metadata correlate well with external datasets. The MAG also has very good coverage across different domains with a slight bias towards technical disciplines. On the other hand, there are certain limitations to completeness. Only 37 million papers out of 1.26 billion have some citation data. While these papers have a good mean total citation count that is consistent with expectations, there is some level of noise when extreme values are considered.\\
Keywords: Scholarly Communication; Publication Datasets; Open Citation Data; Data Mining;\\
Research Evaluation