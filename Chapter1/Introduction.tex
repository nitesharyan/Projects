%\chapter{synopsis}
%\label{chap:intro}

\graphicspath{{Chapter1/Figures/}}



%\chapter{Introduction}
\chapter{Longitudinal Analysis of Microsoft Academic Graph (LAoMAG)}


\section{Introduction}
\label{S:1}

Citation is a key token of recognition that is used to pay tribute to pioneers for their breakthrough discoveries, credit to peers by acknowledging related works, proof reading and improving methodologies to correct one’s own work or other related works to extend research in a new dimension. Besides, it is a fundamental quantifier of quality assessment of several scientific entities at an atomic level and recognizes an author’s research potential in subsequent related works in same or interdisciplinary field of research. Citation analysis in terms of measuring, computing, quantifying, comparing entities, analyzing and re-defining metrics is perhaps, a significant tool to depict progress of science in society over the years.\\
Thus, a citation network represents research inclination and knowledge graph of science in which research output in terms of scientific publications, author or venue form knowledge sources or nodes and their interlinked relationships mapped in terms of citation represents relatedness among various aspects of knowledge. The complexity of citation network has increased ten-fold since 1990’s with exponential increase in node and edge count and its rapid changing dynamics over time. This complexity is getting so specialized that an exhaustive study is required to analyze trends and make crucial decisions in order to maintain research standard. In citation network, when a paper (Pi) gets cited by another paper (Pj), key entity author (Ai) is in turn getting cited by an author (Aj).\\
Broadly, a citation is a reference to a published or unpublished node (paper, author or venue) to acknowledge contribution made by the cited node. More precisely, a citation is an abridged remark added to an intellectual research work made by an author in form of a direct entry in the bibliographic reference section of current work at hand hence, acknowledging its significance. In academic publishing, it is used as a token of recognition to pay homage to pioneers or identify work of peers or others concerning topic of discussion where citation appears. In general, citation is a combination of in-body citation and bibliographic entry.\\
Microsoft Academic graph is one of the largest citation dataset available as an open source for research works. In this dataset we have, a broad   classification of Bibliometrics and citation given to each and every pioneer in order to keep a record of their work. 

{\em Motivation:} Uniformity in author’s contribution measure to measure author’s impact in the academic community, citations should not be considered the only source of measurement. Social bonding of authors through multi-facet relationships like co-authorship, co-publishing (publishing to the same venues), co-chairing (in conferences), co-editorship (of journals), co-affiliation (in academic/research institutes), co-citation (of papers), and so on also bears significant weightage to determine a fair ranking strategy for an author. Besides, h-index and g-index capture it her growth or saturation of scientific success of authors. They therefore, fail to capture decline of success.\\
Citation profiling of papers into common patterns for different fields of science, finite patterns from citation analysis have been detected. The motivation of our current research begins from a fundamental question that ‘Are distributions in citations very identical for different fields of science or rather contrasting? A common accord in the literature reveals that citation dynamics seems to follow generic universal pattern. Modelling heterogeneous citation patterns on a temporal scale, it is found that citation distribution seems to fit some of the generic scaling laws like exponential power law model and log normal behaviour. For an accurate estimation of predicting future citations, we evaluate some of the imperceptible aspects like initial attractiveness or increase in likelihood factor generating to linear preferential attachment models.\\
Anomalous pattern detection in journals: Measuring influence of social networks in scholarly communication with heightened importance of journal metrics such as impact factor, one fundamental question that arises is- what is the ethically admissible limit of citation flow within a small community of researchers? Also, what should be the pre-set norms of an ethical publication process? Since, most of these metrics evaluate only based on citation count, it demands researchers and journals to constantly improve their research standard through authentic contributions. For detecting anomalous citation patterns, a detailed research on a macroscopic level of journal-journal citation network, and measurement of biasness in publisher’s network and editor-co authorship networks is required.\\
Impartial review process prior to publication, an unbiased review process is also of utmost necessity. On a macroscopic view, impact of journals is getting tossed up as a result of underlying microscopic network formed between authors, editors, reviewers, publication houses, board of governors etc. For related reasons, publication houses also urge authors towards coercive citation exchange. In some cases, it is found that belonging to same field of research, a publisher publishes a sister journal with an extended up to date research content of its parent journal thus, trying to uplift performance but soon, it disappears. Also, sometimes a journal’s editorial policies are influenced by its parent publisher.\\\\
{\em Objective:} Broad objective of our research is to study impact of several factors in quantifying author’s performance (one who publish), publications (one that is published) as well as the reviewing process (fairness of selection) of accepting a paper for publication. Assessment of science is important for many different reasons. For researchers at an early stage of their careers, a metric of scientific work may provide significant feedback to their progress and their exact position in the scientific world. For the recruitment committees in universities/research institutes, such a metric may simplify the task of wading through bunch of applications to select a list of potential applicants for the interview. For university administrators, these metrics may help to judge researchers seeking promotion or tenure. For the departmental chairs, these metrics may help suggesting annual raises and the allocation of scarce departmental resources. For scientific societies, these metrics may influence selecting award recipients. For research granting agencies, an assessment of scientific fields would help identifying areas of progress and vitality. For legislative bodies and boards of directors, a measure of science may provide a means of documenting performance, ensuring accountability, and evaluating the return on their research investment. Measures of science may have other applications like, identifying the structure of science, the impact of academic journals, influential fields of research in current time, and factors that may contribute to the new discoveries. 


%{\em Contribution:} In our project, we set up a cluster of nodes to work as one parallel computer connected in star network topology in LAN via a switch. We try to analyze performance of several MPI point to point and collective functions (MPI\_Send, MPI\_Receive, MPI\_Scatter, MPI\_Gather, MPI\_Bcast, MPI\_Reduce etc) which are widely used to write large parallel applications. We try to understand lower and upper bounds of performance. We also aim to present a comparative analysis of several MPI functions as to which function performs better on this network with changing number of nodes, processes and message sizes. Overall, it could help to scale communication linearly with number of processes.
 

\section{Related Work}
\label{S:2}
Derek J.de Solla Price in his 1965 publication first emphasized on inherent citation links between a networks of scientific publications. Around the same year, worldwide growing dynamic citation networks is described in one of the works by Ralph Garner. Besides, Garfield and Sher validated that citation analysis could be used for generating topological maps depicting evolutionary growth of scientific topics. Later, in around 2002 an automated software was developed HistCite’ with combined efforts of E. Garfield, A.I. Pudovkin and V.S. Istomin. In order to keep a record of which recent works cite previous articles in the same field and establish a relationship between papers, an indexing of citations in form of bibliographic databases was started being used. E. Garfield in 1960, first introduced citation index for papers that is, ‘Science Citation Index (SCI).’ In 1997, Cite Seer published first automated citation indexing including only Computer Science and Information Science fields. The most widely used citation indices in academic domain include Google Scholar, Microsoft Academic Cite Seer, Scopus published by Elsevier, Web of science by citation indexing giant Thomson Reuters etc. While some of them are available on line, some can only be collected by paying subscription charges. Other than information retrieval, such citation indexing led to the advent of a new field ‘Bibliometrics’ where, researchers became increasingly focused on measuring research quality. It also, led to popularly used journal performance metric impact factor. These bibliographic databases help to extract citation graphs thus, supplementing and detecting patterns that lead to breakthrough technological discoveries. With growing author, paper and citation rate, citation network is turning into a complex network and such citation index eases the task of segregating into multiple layers and inter-layer mapping between diverse entities.\\

In 1965, a milestone work published by Irving Sherdemon strateda co-relation between frequency of citations collected and eminence where, noble prize winners and their works were cited comparatively 30-50 times more than average researchers. Such evidences claim that these bibliographic data could be used to measure impact of not only journals or papers but also, individual authors, research groups, collaboration influences, institutions, and countries contribution to the academic community. Further, such citation impact indicators could help in ranking authors measuring based on their true credentials.\\

We do an extensive survey on existing literature done with citation analysis and divide it into three categories- Author centric, paper centric and venue centric related works.

{\em Methodology:}
On a macroscopic view, these data dictionaries represent citation relationships (edges) between chosen entities under study and hence, forms a complex citation network. These dictionaries represent mappings such as paper id-year, paper id-references, author id-paper id, paper id-list of author id, author id-author name, field id-field name, author id-list of co-authors id, conference id-conference name, journal id-journal name, paper id-conference/journal id etc. Thus, we propose modelling such citation mappings in form of single or multi-layer network models as suited analysing problems. For author centric study addressing deficits of present performance metrics using page rank based algorithm, we use multi layered model with multiple features added onto each layer. Using such citation mappings, we create network using I-Graph module. Various citation network data set is available online to explore, among them are net miner is quite popular for its open access but comparatively smaller than Microsoft Academic Search Bibliography dataset. According to our requirement we develop two different datasets also. One is for reviewer assignment problem as no such datasets available for open access and the other is pure citation dataset with some extra fields according to our requirement.